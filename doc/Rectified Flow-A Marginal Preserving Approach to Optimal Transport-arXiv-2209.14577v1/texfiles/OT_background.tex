

\paragraph{Notation} 
Let $\Cone$ be the set of continuously  differentiable functions $f\colon \RR^d\to \RR$, and $\Cc$ the functions in $\Cone$ whose support is compact. 
For a time-dependent velocity field $v \colon \RR^d\times[0,1]\to \RR$, 
we write $v_t(\cdot) = v(x,t)$ and 
use $\dot v_t(x) \defeq \partial v(x,t)$ and $\dd v_t(x) \defeq \partial_x v(x,t)$ to denote the partial derivative w.r.t. time $t$ and variable $x$, respectively. 
We denote by $C^{2,1}(\RR^d\times [0,1])$
the set of functions $f\colon \RR^d\times [0,1]\to \RR$ that are second-order continuously differentiable w.r.t. $x$ and first-order continuously differentiable w.r.t. $t$. 
In this work, 
an ordinary differential equation 
(ODE) $\d z_t = v_t(z_t) \d t$ should be interpolated as an integral equation $z_t = z_0 + \int_0^t v_t(z_t) \dt $. 
For $x\in \RR^d$, $\norm{x}$ denotes the Euclidean norm. We always write $c^*$ as the convex conjugate of $c\colon \RR^d\to \RR$, that is, $c^*(x) = \sup_{y\in\RR^d} \{x\tt y - c(y)\}$. 

Random variables are capitalized (e.g., $X,Y,Z$) to  distinguish them with deterministic values (e.g, $x,y,z$). 
Recall that an $\RR^d$-valued random variable  
$X=X(\omega)$ is a measurable function $X\colon \Omega \to \RR^d$, where $\Omega$ is an underlying sample space equipped with a $\sigma$-algebra $\mathcal F$ and a probability measure $\meas P$. 
The triplet $(\Omega, \mathcal F, \meas P)$ form the underlying probability space, which is omitted in writing in the most places.  
We use $\law(X)$ to denote the probability law of $X$, which is the probability measure $\meas L$ that satisfies $\meas L(B) = \meas P(\{\omega \colon X(\omega) \in B\})$ for all measurable sets on $\RR^d$. 
For a functional $F(X)$ of a random variable $X$, 
the optimization problem $\min_{X}F(X)$ %
technically means to 
find a measurable function $X(\omega)$ to minimize $F$, even though we omit the underlying sample space $\Omega$. When $F(X)$ depends on $X$ only through $\law(X)$, the optimization problem is equivalent to finding the optimal $\law(X)$. 




\paragraph{Outline} The rest of the work is organized as follows. Section~\ref{sec:ot} 
introduces the background of optimal transport. Section~\ref{sec:rectopt} 
reviews rectified flow of \cite{rectified} from an optimization-based view. 
Section~\ref{sec:marginal0} characterizes 
the if and only if condition 
for two differentiable stochastic processes to have equal marginal laws. 
Section~\ref{sec:crectify} introduces the main $c$-rectified flow method and establishes its theoretical properties. 
 


\section{Background of Optimal Transport}
\label{sec:ot}
This section introduces the background of optimal transport (OT), 
including both the static and dynamic formulations.  
Of special importance is the dynamic formulation, 
which is closely related to the rectified flow approach. 
The readers can find systematic introductions to OT in 
a collection of excellent textbooks 
\cite{villani2021topics, figalli2021invitation,   ambrosio2021lectures, peyre2019computational, ollivier2014optimal,
santambrogio2015optimal,
villani2009optimal}.  

\paragraph{Static formulations} 
The optimal transport problem was 
first formulated by 
Gaspard Monge in 1781 when he studied 
 the problem of how to 
 redistribute mass, e.g., a pile of soil, with minimal effort. 
 Monge's problem can be formulated as
 \bbb \label{equ:m}
 \inf_{T} \e{c(T(X_0)- X_0)} ~~~~s.t.~~~~ \law(T(X_0))=\tg_1, ~~~ \law(X_0)=\tg_0,
 \eee 
 where we minimize the $c$-transport cost 
 in the set of deterministic couplings $(X_0,X_1)$ that satisfy $X_1 = T(X_0)$ for a transport mapping $T\colon \RR^d\to \RR^d$. 
The Monge–Kantorovich (MK) problem in \eqref{equ:mk} is the relaxation 
of \eqref{equ:m} to the set of all (deterministic and stochastic) couplings of $\tgg$.  
The two problems are equivalent when the optimum of \eqref{equ:mk} is achieved by a 
deterministic coupling, which is guaranteed if $\tg_0$ is  an absolutely continuous measure on $\RR^d$. 

A key feature of the MK problem is that it is
a linear programming w.r.t. the law of the coupling $(X_0,X_1)$, and yields a  dual problem of form: %
\bbb \label{equ:dmk}
\sup_{\mu, \nu} 
\tg_1(\mu)  - \tg_0(\nu) 
~~s.t.~~
 \mu(x_1) - 
\nu(x_0)\leq c(x_1-x_0),~~~~\forall (x_0,x_1),
\eee 
where we write $\tg_1(\mu) \defeq \int \mu(x) \d \tg_1(x)$, and $\mu,\nu$ are optimized in all functions from $\RR^d$ to $\RR$. 
For any coupling $(X_0,X_1)$ of $\tgg$, and $(\mu,\nu)$ satisfying the constraint in \eqref{equ:dmk}, it is easy to see that 
\bbb \label{equ:dualderive}
\E[c(X_1-X_0)] 
\geq \E[\mu(X_1) - \nu(X_0)] 
= \tg_1(\mu) - \tg_0(\nu). 
\eee 
As the left side of \eqref{equ:dualderive} 
only depends on $(X_0,X_1)$ and the right side only on $(\mu,\nu)$, one can show that $(X_0,X_1)$ is $c$-optimal  and $(\mu,\nu)$ solves \eqref{equ:dmk} iff 
$\mu(X_0) + \nu(X_1) = c(X_1-X_0)$ 
holds with probability one, which provides a basic optimality criterion. %
Many existing 
OT algorithms are developed by exploiting the primal dual relation of \eqref{equ:mk} and 
\eqref{equ:dmk} 
(see e.g., \cite{korotin2022neural}), but have the drawback of yielding  minimax problems that are challenging to solve in practice. %
 
If $c$ is strictly convex, 
the %
optimal transport map of \eqref{equ:m} 
is unique (almost surely) and yields a form of 
\bb %
T(x)=x + \dd c^*(\dd \nu(x)), ~~~~
\ee 
where $c^*$ is the convex conjugate function of $c$, and $\nu$ is an optimal solution of \eqref{equ:dmk}, which is  $c$-convex in that $\nu(x) =  \sup_{y}\left \{ -c(y-x) +  \mu(y)\right \}$ with $\mu$ the associated solution. 
In the canonical case of quadratic cost  $c(x) = \frac{1}{2}\norm{x}^2$, we can write $T(x) = \dd \phi(x)$, where $\phi(x) \defeq \frac{1}{2}\norm{x}^2 + \nu(x)$ is a convex function. 



\paragraph{Dynamic  formulations} 
Both the MK and Monge problems can be equivalently framed in dynamic ways 
as finding continuous-time processes that transfer $\tg_0$ to $\tg_1$.  
Let $\{x_t \colon t\in[0,1]\}$ be a smooth path connecting $x_0$ and $x_1$, whose time derivative is denoted as $\dot x_t$. 
For convex $c$, by Jensen's inequality, we can represent the cost $c(x_1-x_0)$ in an integral form: 
$$
c({x_1 - x_0}) = c\left (\int_0^1 \dot x_t \dt \right) =  \inf_{x} \int_0^1 c({\dot x_t}) \dt,
$$
where the infimum is attained when $x_t$ is the  linear interpolation (geodesic) path:  $x_t = t x_1 + (1-t)x_0$. 
Hence, the MK optimal transport problem \eqref{equ:mk} is equivalent to 
\bbb \label{equ:qt}
\inf_{\vv X}  
\E\left  [ \int_0^1 c(\dot X_t) \dt \right ]  ~~~~~s.t.~~~~ \law(X_0) = \tg_0, ~~\law(X_1) = \tg_1,
\eee 
where we optimize in the set of  time-differentiable %
stochastic processes 
$\vv X = \{X_t \colon t\in[0,1]\}$.
The optimum of \eqref{equ:qt} 
is attained by  $X_t = t X_1 + (1-t)X_0$ when
 $(X_0,X_1)$ is a $c$-optimal coupling of \eqref{equ:mk}, which is known as the \emph{displacement interpolation}  \citep{mccann1997convexity}. 
 We call the objective function in \eqref{equ:qt} the path-wise $c$-transport cost. 

The Monge problem can also be framed in a dynamic way. 
Assume the transport map $T$ can be induced by an ODE model 
$\d X_t = v_t(X_t)\dt $ such that $X_1 = T(X_0)$. Then the Monge problem is equivalent to %
\bbb \label{equ:cm}
\inf_{v,\traj X} \E\left [ \int _0^1  c({v_t(X_t)}) \dt  \right] ~~~~~s.t.~~~~~ \d X_t = v_t(X_t)\dt ,~~~~~
\law(X_0) = \tg_0, ~~~~~ \law(X_1) = \tg_1, 
\eee 
which is equivalent to restricting $\vv X$  in \eqref{equ:qt} to the set of processes that can be induced by ODEs. 

Assume that $X_t$ following  $\d X_t = v_t(X_t)\dt $ yields a density function $\varrho_t$. Then 
it is well known that $\varrho_t$ satisfies the continuity equation:  %
$$
\dot \varrho_t + \div(v_t\varrho_t) = 0.
$$
Hence, we can rewrite \eqref{equ:cm} into an optimization problem on $(v, \varrho)$, yielding the celebrated \emph{{\bbformula} formula} \cite{benamou2000computational}: 
\bbb \label{equ:bb} 
\inf_{v, \varrho} 
\int_0^1  \int c(v_t(x))  \varrho_t(x)\d x \dt  
~~~~s.t.~~~~ 
\dot  \varrho_t  + \div(v_t \varrho_t) = 0,~~~~\rho_0  = \d\tg_0/\dx, ~~~~\rho_1  = \d \tg_1/\dx, \eee 
where $\d \pi_i/\d x$ denotes the density function of $\tg_i$. 
The key idea of 
\eqref{equ:cm} and \eqref{equ:bb} is to 
restrict the optimization of \eqref{equ:qt} to the set of deterministic processed induced by ODEs, which significantly reduces  the search space. 
Intuitively, 
Jensen's inequality $\E[c(Z)]\geq c(\E[Z])$ shows that we should be able to reduce the expected cost of a stochastic process 
by ``marginalizing'' out the  randomness. 
In fact, 
we will show that, for a differentiable stochastic process $\vv X$, %
its ($c$-)rectified flow yields no larger 
path-wise $c$-transport cost in \eqref{equ:qt} than $\vv X$ (see Lemma~\ref{thm:rectdual} and Theorem~\ref{thm:optmultimarg}).  

However, all the dynamic formulations above are still highly challenging to solve in practice.  
We will show that $c$-rectified flow can be viewed as a special coordinate descent like  approach 
to solving \eqref{equ:qt} (Section~\ref{sec:crectifyOptView}). %

